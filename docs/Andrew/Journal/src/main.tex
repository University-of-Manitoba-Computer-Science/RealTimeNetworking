\documentclass{article}
\usepackage{graphicx} % Required for inserting images
\usepackage{url}
\title{4060 - T29: Real Time Networking \\ Student Log}

\author{Andrew Marinic - 7675509}
\date{January - April 2024}

\begin{document}

\maketitle
\newpage

\tableofcontents

\newpage

\section{Introduction}
\paragraph{Intro}
The following is a log or perhaps a journal of my efforts in COMP 4060 - T29 - Real Time Networking. I will break the document into sections for each Month and subsections for each week. Each day will contain a paragraph explaining what I had accomplished. 
\section{January}
\subsection{January 8-12}
\paragraph{Jan 8}
I did some preliminary reading of the documents provided on CAN networks. I only spent about an hour reading, little notes. 
\paragraph{Jan 11}
Dug into the data sheets. Started examining the UART drivers we need to build. Probably 30 mins or so taken aside. Industrial Project has hit the fan a bit and taken up more time. I set up a calendar for the group as we all have fairly tight schedules right now.
\subsection{January 15-19}
\paragraph{Jan 15}
We had an hour-long group meeting today. We discussed the pace of the project we would like to achieve. Send off an email to confirm some items and inform the Supervisor of our meeting availability. I started to dig into blinky again with the updated code. To my grief, I have switched to VScode and started to try and get the debugging working with the help of the other group members. I spent another hour or so getting the debugging to work in VScode, I stopped once I had the debugger open, openOCD via VScode. I started to get a little frustrated and had to finish my industrial project proposal so I stopped for the evening. I will discuss with the group for advice and continue later this week. 
\paragraph{Jan 17}
Today I spent around 4 hours deep-diving CAN and making notes on it. Then I proceeded to try and get debugging working one last time before asking for help. If I cannot accomplish this by, Friday I will reach out. I spent approx 1.5 hours on this, but more time was spent reading about it than coding. 

\paragraph{Jan 18}
Spent approximately 1 hour reading into the Curiosity Nano's CAN controller. I think I understand the requirements for setting up and how the INIT and CONFIG registers work together. 
\paragraph{JAN 22}
I spent about an hour and a half finishing my CAN notes which are visible in the Notes section. I have done some looking into coding the CAN controller of the curiosity nano. 
\begin{itemize}
    \item \url{https://github.com/MikroElektronika/mikrosdk_click_v2/tree/master/clicks/canfd3} Here is the CAN FD 3 example code. I gave it a quick browse for something valuable but was mainly looking for CAN controller init code keeping it in my back pocket  though
    \item \url{https://microchip-mplab-harmony.github.io/reference_apps/apps/sam_e51_cnano/same51n_mikroe_click/readme.html} I also gave themse projects a quick brows for the same as above, but I don't think any use the CAN controller just UART. 
    \item \url{https://microchip-mplab-harmony.github.io/reference_apps/apps/sam_e54_xpro/same54_can_usb_bridge/readme.html#hardware-used} I do however think there is something useful here. It uses the same micro controller so it should have INIT, read/write code, ect. This will take some digging though. 
    Today I put in approximately 3.5 hours into researching and documenting. 
\end{itemize}
\paragraph{JAN 24}
I tinkered more with debugging for about an hour or two this morning. It seems to be somewhat working but I am still not getting proper debugging and have no output from the debug calls. I will give it one last go. I think we will plan a group in person meeting, maybe input from the other students will help. I maybe just need to keep fresh eyes on it and stop getting frustrated and understand it better, but config files are frustrating to me. In a more productive note, I have started some framework for my drive code. I am going to clean up some of my I2C code for my gyro so I have a device to poll data from and send over the CAN network. This way I can make sure I understand the new systimer and clock speed a little better with something I understand. I suspect I will have to use some prescalers to compensate for the 60 times faster clock. I will do a commit at end of day and upload this as a pdf. 
\section{NOTES}

\subsection{CAN Overview}
\paragraph{Controller Area Network (CAN)} is a standard for micro-controller and device communication. It uses messages, and was originally designed as a multiplexing method. The physical layer utilises twisted pairs (CAN+ and CAN-). Messages are framed with IDs which dictate priority. Logical 0 is Dominant and logical 1 is recessive. This means IDs that have larger values (1's in high bit places) are lower priority. Arbitration is done via first bit with a 1. All nodes see transmission.
\paragraph{Nodes} All nodes can send and receive, but not at the same time. The priority is determined by the frame ID. Messages are transmitted using non-return-to-zero (NRZ) format \\ All nodes require the following: 
\begin{itemize}
    \item Some controller :CPU/Microprocessor/host processor/micro controller
        \begin{itemize}
            \item Decides what messages mean and what to transmit
            \item Handles talking to other devices
        \end{itemize}
    \item CAN Controller
        \begin{itemize}
            \item Receiving: Stores bits until entire message is received, then can trigger interrupt for retrieval
            \item Transmitting: If the host, can send messages via CAN controller in a serial manner.
        \end{itemize}
    \item Transceiver (ISO 11898-2/3) Medium Access Unit (MAU)
        \begin{itemize}
            \item Receiving: converts at the CAN bus level for CAN controller use, protective layer for CAN controller. 
            \item Transmitting: converts bit stream from CAN controller to the CAN bus
        \end{itemize}
\end{itemize}



\subsection{CAN: IDs and arbitration}
\paragraph{ID Arbitration} When nodes transmit they see all messages including their own. If they transmit a 1 and see 0 they quit and lose arbitration. This is because 0 is dominant and they then know they do not have priority. Because of this, whenever there is a collision the lower ID will win. When a collision does happen.
The recessive message waits for the dominant message + 6-bit clocks then
attempts again
This means the first frame to transmit a 1 is the loser, thus highest
priority id frame is all 0s followed by 00\ldots.001
\paragraph{IDs as priorities}
Using ID for the type of data, or the sending node ignores the fact ID is
also used as message priority. This leads to poor real-time performance.
CAN bus is limited to around 30\% to ensure deadlines if you don't build
around the priority. Otherwise, you can achieve 70 to 80\% CAN bus usage and
have reliable deadlines.

\subsection{CAN: Bit timing}
\paragraph{Nominal Bit Time:} Time it takes to send bit components:\\

\paragraph{Synchronization} Synchronization is important to the CAN protocol. It prevents errors and allows for arbitration's to occur. Recolonization occurs every single recessive to dominant transmission during the frame. In order to sync the nominal bit time is segmented into quanta and then certain aspects can be altered to allow for synchronization. 
The nominal bit time is broken down in the following way. Each are assigned a number of quanta. For example a system where we break our nominal bit time into 10 quanta.


Sync (1 quanta)

Propagation (3 quanta)

Phase segment 1 (3 quanta)
 
Phase segment 2 (3 quanta)


Synchronization occurs as follows. 

\begin{enumerate}
\def\labelenumi{\arabic{enumi}.}
\item
  \textbf{Bus Idle} -\textgreater{} wait for first recessive to dominant
  transition
\item
 \textbf{ Hard synchronization}
\item
  \textbf{Resync} occurs on every recessive to dominant transition during the
  frame (message?)

  \begin{enumerate}
  \def\labelenumii{\alph{enumii}.}
  \item
    CAN controller expects this at multiple of nominal bit time.
  \item
    Else It adjust nominal bit time accordingly.
  \end{enumerate}
\end{enumerate}

~

\paragraph{Resync and Adjustment process:}

\begin{itemize}
\item
  Produce a number of quanta to divide the bits\textquotesingle{}
  segments into time slices.

  \begin{itemize}
  \item
    The number of quanta can vary based on the controller
  \item
    The quantity of quanta a segment is assigned can vary depending on
    system needs
  \end{itemize}
\item
  On out-of-sync (before or after) transition controller calculates the time
  difference, to compensate:

  \begin{itemize}
  \item
    If we need to lengthen we do so to phase 1
  \item
    If we need to reduce time we do so in phase 2.
  \end{itemize}
\item
  As a result of either a or b, we have adjusted the timing of the
  receiver to the transmitting node and synchronized them.
\item
  We continuously do so at every recessive to dominant transition to
  keep synchronization

  \begin{itemize}
  \item
    This reduces errors induced by noise (random error)
  \item
    Allows for resync to nodes that lost arbitration back to the one
    that won previously.
  \item
  \end{itemize}
\end{itemize}

\subsection{CAN Protocol Layers:}
\begin{itemize}
    \item Application layer
    \item Object layer
    \item Transfer layer
    \item Physical layer
\end{itemize}


We are building the transfer layer?

~

\subsubsection{CAN Transfer layer:}

\begin{itemize}
\item
  Most of the CAN standard applies to this layer, it is what receives
  messages from the physical layer and into the object layer for use in
  the application.
\item
  Transfer layer is responsible for:

  \begin{itemize}
  \item
    Synchronization
  \item
    Bit timing
  \item
    Message framing
  \item
    Arbitration
  \item
    Acknowledgement
  \item
    Error Detection
  \item
    Signalling
  \item
    Confinement
  \end{itemize}
\item
  To accomplish the previous responsibilities, it performs the following
  tasks:

  \begin{itemize}
  \item
    Fault confinement
  \item
    Error detection
  \item
    Message Validation
  \item
    Arbitration
  \item
    Message framing
  \item
    Transfer rate and timing
  \item
    Information routing
  \end{itemize}
\end{itemize}

~

Physical layer

\begin{itemize}
\item
  Pinout:

  \begin{itemize}
  \item
    Pin 2: CAN- (Low)
  \item
    Pin 3: GND
  \item
    Pin 7: CAN+ (High)
  \item
    Pin 9: CAN V+ (power)
  \end{itemize}
\end{itemize}

\begin{quote}
~
\end{quote}

\subsection{CAN Frames:}

\begin{itemize}
\item
  Two types of frame format

  \begin{enumerate}
  \def\labelenumi{\arabic{enumi}.}
  \item
    Base frame format

    \begin{enumerate}
    \def\labelenumii{\roman{enumii}.}
    \item
      11- bits for identifier
    \item
      IDE bit dominant
    \end{enumerate}
  \item
    Extended frame format

    \begin{enumerate}
    \def\labelenumii{\roman{enumii}.}
    \item
      11- bit identifier + 18-bit extension = 29-bit identifier
    \item
      IDE bit recessive
    \end{enumerate}
  \end{enumerate}
\item
  There are four types of frames.
\item
  Regardless of type all begin with a start of frame (SOF) bit to signal
  start of frame transmission.
\item
  Frame types:

  \begin{enumerate}
  \def\labelenumi{\arabic{enumi}.}
  \item
    Data Frame: a frame containing node data for transmission
    .

  \item
    Remote frame: requests transmission of an identifier
  \item
    Error frame: frame type for any node detecting an error.
  \item
    Overload frame: a buffer/delay for data or remote frame.
  \end{enumerate}
\end{itemize}
\begin{figure}
    \centering
    \includegraphics[width=1\linewidth]{2880px-CAN_Bit_Timing2.svg.png}
    \caption{CAN Bit Timing: Wikipedia}
    \label{fig:enter-label}
\end{figure}
\begin{figure}
      \centering
      \includegraphics[width=1\linewidth]{Dataframe.png}
      \caption{Data frame: Wikipedia}
      \label{fig:enter-label}
  \end{figure}





\end{document}